\chapter{Laboratório de Engenharia Química}

Informações e Resumos sobre a matéria.

\section{Ementa}

Ementa da disciplina:


\begin{itemize}


\item Utilização de instrumentos de medida de vazão. 
\item Perda de carga em tubulações eacidentes. 
\item Características e seleção de bombas hidráulicas e sopradores. 
\item Cuidadosoperacionais e rendimento térmico de caldeiras flamatubulares. 
\item Trocadores de calor e condensadores. 
\item Operação e parâmetros de projeto de filtros-prensa. 
\item Leitos fluidizados. 
\item Equilíbrio líquido-vapor. 
\item Cinética Química. 
\item Destilação. 
\item Identificação da dinâmica de processos I. 
\item Equilíbrio de fases.



\end{itemize}




Conteúdo Programático:
\begin{itemize}
\item  Utilização de instrumentos industriais de medida de vazão. Revisão teórica.
Rotâmetros, Placas de Orifícios e Venturis. Determinação da curva de calibração.
Análise dos resultados experimentais e comparação dos parâmetros obtidos com os
encontrados na literatura. Métodos de projeto desses instrumentos. (3 h)
\item  Quantificação de perda de carga distribuída e localizada. Revisão teórica. Aplicação
em projetos de tubulações. Análise de resultados experimentais e comparação dos
parâmetros estimados com os correspondentes disponíveis na literatura. (6 h)
\item  Bombas hidráulicas. Revisão teórica. Tipos, características operacionais e seleção.
Carga líquida positiva na sucção. Construção das curvas características de uma
bomba centrífuga. (6 h)
\item Separação sólido-líquido em sedimentador contínuo. Teste de proveta. Cálculo da
área de um sedimentador industrial. (3 h)
\item  Caldeiras industriais. Revisão teórica. Caldeiras flama-tubulares e aquatubulares.
Procedimentos padronizados de segurança para a operação de caldeiras
flamitubulares. Determinação da eficiência térmica. Métodos para o acompanhamento
da eficiência de queima. (3 h)
\item  Trocadores de calor. Revisão teórica. Metodologia de projeto. Determinação
experimental e teórica do coeficiente global de transferência de calor. Análise dos
resultados. Comparação do desempenho de uma unidade de módulos bitubulares em
função da configuração do escoamento - Paralelo e Contra-Corrente. (6 h)
\item Condensadores. Revisão teórica. Metodologia de projeto de condensadores de
substâncias puras. Estimativa do consumo de vapor em condições operacionais préestabelecidas. Comparação com resultados experimentais. Determinação
experimental e teórica do coeficiente global de transferência de calor. (3 h)
\item  Separação sólido-líquido em filtro-prensa piloto. Revisão teórica. Aplicação da teoria
simplificada da filtração à pressão constante com formação de torta compressível, na
determinação dos parâmetros de projeto e "scale-up". (6 h)
\item  Fluidização. Revisão teórica. Determinação da velocidade e porosidade mínimas de
fluidização e da queda de pressão de leitos fluidizados a gás. Demonstração da
histerese de fluidização. (3 h)
\item  Equilíbrio Líquido-Vapor. Revisão teórica. Determinação do diagrama de equilíbrio
de sistemas binários, miscíveis em todas as proporções, utilizando um aparelho
Othmer modificado. Concentrações medidas em fase líquida através do índice de
refração. Teste de consistência dos resultados, usando-se modelos de Margules e
Wilson. (6 h)
\item Cinética Química. Revisão teórica. Calibração do sistema reacional - Temperatura,
Vazão e Concentração. Influência da concentração e temperatura. Análise dos
produtos da reação.Análise quantitativa: aplicação dos métodos integrais, diferenciais
e tempo de meia vida. Ordem de reação e energia de ativação. (6 h)
\item  Destilação. Revisão teórica. Partida e operação em refluxo total de uma torre de
destilação de pratos com borbulhadores, em escala piloto. Levantamento de
parâmetros operacionais. (3 h)
\item  Identificação da Dinâmica de Processos. Levantamento de dados experimentais
em regime transiente e identificação de modelos dinâmicos para a sua descrição. (3
h)
\item  Equilíbrio de fases com o uso de um espectofotômetro. Confecção da curva de
calibração. Determinação da constante de distribuição pela medida da absorbância
das fases em equilíbrio. Avaliação dos modelos para o coeficiente de atividade. (3 h) 
\end{itemize}

\section{Bibliografia}

\begin{itemize}
\item  Perry,R.H. e Green,D.W. (ed.) "Perry's Chemical Engineer's Handbook". 7th edition,
McGraw-Hill, New York, 1997.
\item  McCabe, W.L., Smith, J.C., Harriot, P. “Unit Operations of Chemical Engineering”.
McGraw-Hill International Editions, 4th Ed., New York, 1985.
\item  Peçanha, R. P.: “Sistemas Particulados \& Operações Unitárias Envolvendo
Partículas e Sólidos”, Elsevier – Campus, Rio de Janeiro, 2014				
\end{itemize}


Bibliografia Complementar ( no mínimo 5)
1. Incropera, F.P., DeWitt, D.P., Bergman, T. L., Lavine, A. S. (2014) Fundamentos de
Transferência de Calor e de Massa. 7a
Edição. LTC Livros Técnicos, Rio de Janeiro.
2. Massarani, G. “Fluidodinâmica em Sistemas Particulados”, 2ª edição, E-Papers,
2002.
3. J.D. Seader, E. Henley, Separation Process Principles, John Wiley, New
York, 2nd Edition, 2005
4. Russel, T. W. F., Robinson, A. S., Wagner, N. J., “Mass and Heat Transfer -
Analysis of mass contactors and heat exchangers”, Cambridge University Press, 2008.
5. Pinho, M.N. Fundamentos de Transferência de Massa (2008).Editora: Martins
Fontes - Selo Martins, Ist Press, 1ª Ed.



\section{Aulas Experimentais}

CARACTERÍSTICAS DAS AULAS PRÁTICAS: Obtenção de dados experimentais em equipamentos localizados no LADEQ. Elaboração de relatórios e pequenos projetos.

\begin{itemize}
	\item Destilação Etanol-Água batelada.
	\item Definição de Parâmetros Cinéticos
	\item Filtro Prensa
	\item Equilíbrio de fases
	\item 
\end{itemize}

\section{Cheat Sheet}
%\begin{figure}[h]